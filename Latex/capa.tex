\documentclass[
% -- opções da classe memoir --
12pt,				% tamanho da fonte
openright,			% capítulos começam em pág ímpar (insere página vazia caso preciso)
oneside,			% para impressão em recto e verso. Oposto a oneside
a4paper,			% tamanho do papel. 
% -- opções da classe abntex2 --
%chapter=TITLE,		% títulos de capítulos convertidos em letras maiúsculas
%section=TITLE,		% títulos de seções convertidos em letras maiúsculas
%subsection=TITLE,	% títulos de subseções convertidos em letras maiúsculas
%subsubsection=TITLE,% títulos de subsubseções convertidos em letras maiúsculas
% -- opções do pacote babel --
english,			% idioma adicional para hifenização
french,				% idioma adicional para hifenização
spanish,			% idioma adicional para hifenização
brazil,				% o último idioma é o principal do documento
]{abntex2}


% ---
% PACOTES
% ---

% ---
% Pacotes fundamentais 
% ---
\usepackage{lmodern}			% Usa a fonte Latin Modern
\usepackage[T1]{fontenc}		% Selecao de codigos de fonte.
\usepackage[utf8]{inputenc}		% Codificacao do documento (conversão automática dos acentos)
\usepackage{indentfirst}		% Indenta o primeiro parágrafo de cada seção.
\usepackage{color}				% Controle das cores
\usepackage{graphicx}			% Inclusão de gráficos
\usepackage{microtype} 			% para melhorias de justificação
\usepackage{placeins}
% ---

% ---
% Pacotes adicionais, usados no anexo do modelo de folha de identificação
% ---
\usepackage{multicol}
\usepackage{multirow}
% ---

% ---
% Pacotes adicionais, usados apenas no âmbito do Modelo Canônico do abnteX2
% ---
\usepackage{lipsum}				% para geração de dummy text
% ---

% ---
% Pacotes de citações
% ---
\usepackage[brazilian,hyperpageref]{backref}	 % Paginas com as citações na bibl
\usepackage[alf]{abntex2cite}	% Citações padrão ABNT

% --- 
% CONFIGURAÇÕES DE PACOTES
% --- 

% ---
% Configurações do pacote backref
% Usado sem a opção hyperpageref de backref
\renewcommand{\backrefpagesname}{Citado na(s) página(s):~}
% Texto padrão antes do número das páginas
\renewcommand{\backref}{}
% Define os textos da citação
\renewcommand*{\backrefalt}[4]{
	\ifcase #1 %
	Nenhuma citação no texto.%
	\or
	Citado na página #2.%
	\else
	Citado #1 vezes nas páginas #2.%
	\fi}%
% ---
%redefine a capa
\renewcommand{\imprimircapa}{%
	\begin{capa}%
		\center
		\ABNTEXchapterfont\Large \textbf{UNIVERSIDADE FEDERAL DE SERGIPE}
		\\
		\vspace*{1cm}
		{\ABNTEXchapterfont\large\imprimirautor}
		\vfill
		\begin{center}
			\ABNTEXchapterfont\bfseries\LARGE\imprimirtitulo
		\end{center}
		\vfill
		\large\imprimirlocal \\
		\large\imprimirdata
		\vspace*{1cm}
	\end{capa}
}
% ---
% Informações de dados para CAPA e FOLHA DE ROSTO
% ---
\titulo{Aplicação dos Algoritmos de Dijkstra e de Floyd-Warshall no software MATLAB}
\autor{Iuri rodrigo ferreira Alves da silva\\Gregory Medeiros Melgaço Pereira\\Raul Rodrigo Silva de Andrade \\ Rafael Castro Nunes \\ Ruan Robert Bispo dos Santos \\ Vítor do Bomfim Almeida Carvalho}
\local{São Cristóvão,SE}
\data{\today}
\instituicao{%
	Universidade Federal De Sergipe
	\par
	Faculdade de Engenharia Eletrônica
	\par
	Redes e Comunicações}
\tipotrabalho{Relatório técnico}
% O preambulo deve conter o tipo do trabalho, o objetivo, 
% o nome da instituição e a área de concentração 
\preambulo{Relatório em conformidade com as normas ABNT}
% ---

% ---
% Configurações de aparência do PDF final

% alterando o aspecto da cor azul
\definecolor{blue}{RGB}{41,5,195}

% informações do PDF
\makeatletter
\hypersetup{
	%pagebackref=true,
	pdftitle={\@title}, 
	pdfauthor={\@author},
	pdfsubject={\imprimirpreambulo},
	pdfcreator={LaTeX with abnTeX2},
	pdfkeywords={abnt}{latex}{abntex}{abntex2}{relatório técnico}, 
	colorlinks=true,       		% false: boxed links; true: colored links
	linkcolor=blue,          	% color of internal links
	citecolor=blue,        		% color of links to bibliography
	filecolor=magenta,      		% color of file links
	urlcolor=blue,
	bookmarksdepth=4
}
\makeatother
% --- 

% --- 
% Espaçamentos entre linhas e parágrafos 
% --- 

% O tamanho do parágrafo é dado por:
\setlength{\parindent}{1.3cm}

% Controle do espaçamento entre um parágrafo e outro:
\setlength{\parskip}{0.2cm}  % tente também \onelineskip

% ---
% compila o indice
% ---
\makeindex
% ---

% ----
% Início do documento
% ----
\begin{document}
	
	% Seleciona o idioma do documento (conforme pacotes do babel)
	%\selectlanguage{english}
	\selectlanguage{brazil}
	
	% Retira espaço extra obsoleto entre as frases.
	\frenchspacing 
	
	% ----------------------------------------------------------
	% ELEMENTOS PRÉ-TEXTUAIS
	% ----------------------------------------------------------
	
	% ---
	% Capa
	% ---
	\imprimircapa
	% ---
	
	% ---
	% Folha de rosto
	% (o * indica que haverá a ficha bibliográfica)
	% ---
	\imprimirfolhaderosto*
	% ---
	
	% ---
	% Agradecimentos
	% ---
	
	% ---
	
	% RESUMO
	% resumo na língua vernácula (obrigatório)
	\setlength{\absparsep}{18pt} % ajusta o espaçamento dos parágrafos do resumo
	\begin{resumo}
		A busca por uma maior economia e menores custos nas diversas áreas das ações humanas foi um grande ponto de partida para o surgimento da Teoria dos Grafos com o consequente Problema dos Menores Caminhos. Para a resolução desse Problema dos Menores Caminhos surgiram vários tipos de algoritmos, como por exemplo o Algoritmo de Dijkstra e Algoritmo de Floyd-Warshall. Nesse trabalho será feito uma explicação sobre o que são os grafos e sobre os algoritmos utilizados nesse projeto. Após isso, esses algoritmos serão implementados no software computacional MATLAB para ser possível a comparação entre as características obtidas dos dois algoritmos. Com essa comparação será possível dizer qual dos algoritmos é mais eficiente com relação ao número de iterações e tempo de processamento e sob quais condições isso acontece.
		
		\noindent
		\textbf{Palavras-chaves}: Grafos, Problema dos Menores Caminhos, Floyd-Warshall, Dijkstra, Iterações
	\end{resumo}
	% ---
	
	%lista de ilustrações
	% ---
	\pdfbookmark[0]{\listfigurename}{lof}
	\listoffigures*
	\pagebreak
	% ---
	
	%lista de tabelas
	\pdfbookmark[0]{\listtablename}{lot}
	\listoftables*
	\pagebreak
	% ---
	
	%lista de abreviaturas e siglas
	
	
	% sumario
	\pdfbookmark[0]{\contentsname}{toc}
	\tableofcontents*
	\pagebreak
	% ---
	
	%introdução
	\chapter*[Introdução]{Introdução}
\addcontentsline{toc}{chapter}{Introdução}

No mundo globalizado e capitalista que vivemos é cada vez mais necessário caminhos ou métodos para se decidir qual a melhor alternativa ou caminho a ser usado visando uma maior economia, , seja ela na produção do meio industrial, nas conversões de moedas ou até melhores caminhos a serem tomados numa viagem. 

Foi pensando nesses tipos de problemas que surgiu a Teoria dos Grafos em 1736 com Leonhard Euler (1707 – 1783). Um grafo é um conjunto de vértices e um conjunto de arestas que ligam pares de vértices distintos (com nunca mais que uma aresta a ligar qualquer par de vértices).

  Leonhard resolveu um quebra-cabeça local da cidade Königsberg onde vivia, hoje conhecido como O Problema das Pontes de Königsberg. Esse problema consistia em partindo de um ponto inicial, voltar a esse mesmo ponto passando exatamente uma vez por cada ponte da cidade. Euler mostrou que isso só seria possível se cada porção de terra estivesse ligada a um número par de pontes, e , como isso não acontecia na sua cidade, não era possível.
  
    Outro exemplo nesse mesmo sentido é o mostrado em \cite{costalonga2012grafos}, em que um carteiro sempre tentará caminhar a menor distancia possível, assim, para isso acontecer, ele tem de agir de tal forma a passar um número mínimo de vezes em determinada rua, otimizando assim seu trabalho.
  
   Ao longo da história outros problemas similares ao de Euler surgiram, como por exemplo ‘O Problema das Quatro Cores’ . Esse problema surgiu em 1852 e somente em 1976 utilizando métodos computacionais chegou-se a sua solução. 

A Teoria dos Grafos ganhou muita visibilidade já no século passado, pois como dito em \cite{da2011teoria} no desenvolvimento matemático  assim como nas suas aplicações, foi dada grande importância a essa teoria. Com o passar do tempo grafos foram utilizados em várias áreas do conhecimento, desde a Economia até a Biologia. 

Com esse grande destaque, se tornou necessária a criação de algoritmos que agilizassem suas soluções. Entre os mais conhecidos estão o de Dijkstra, de Bellman-Ford e o de Floyd-Warshall.

	
	
	
	\phantompart
	%-
	
	%Revisão Bibliografica
	\include{Referência}
	\phantompart
	%-
	\chapter{Objetivos}
	O principal objetivo desse trabalho foi implementar os algoritmos de Dijkstra e de Floyd-Warshall para ser possível uma comparação do desempenho dos mesmos.
	\include{teorica}
	\phantompart
	\chapter{Formulação do problema}
	\phantompart
	\chapter{Resultados Obtidos}
	\phantompart
	\chapter{Conclusão}
	\phantompart
	
	\cite{algo}
	\cite{cardoso2005teoria}
	\cite{coelho2013teoria}
	\cite{costa2011teoria}
	\cite{costalonga2012grafos}
	\cite{deteoria}
	\cite{feofiloff2011introduccao}
	
	
	\bibliography{biblio}
	%\bibliographystyle{ieeetr}
\end{document}
