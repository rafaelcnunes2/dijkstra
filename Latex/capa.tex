\documentclass[
% -- opções da classe memoir --
12pt,				% tamanho da fonte
openright,			% capítulos começam em pág ímpar (insere página vazia caso preciso)
oneside,			% para impressão em recto e verso. Oposto a oneside
a4paper,			% tamanho do papel. 
% -- opções da classe abntex2 --
%chapter=TITLE,		% títulos de capítulos convertidos em letras maiúsculas
%section=TITLE,		% títulos de seções convertidos em letras maiúsculas
%subsection=TITLE,	% títulos de subseções convertidos em letras maiúsculas
%subsubsection=TITLE,% títulos de subsubseções convertidos em letras maiúsculas
% -- opções do pacote babel --
english,			% idioma adicional para hifenização
french,				% idioma adicional para hifenização
spanish,			% idioma adicional para hifenização
brazil,				% o último idioma é o principal do documento
]{abntex2}


% ---
% PACOTES
% ---

% ---
% Pacotes fundamentais 
% ---
\usepackage{lmodern}			% Usa a fonte Latin Modern
\usepackage[T1]{fontenc}		% Selecao de codigos de fonte.
\usepackage[utf8]{inputenc}		% Codificacao do documento (conversão automática dos acentos)
\usepackage{indentfirst}		% Indenta o primeiro parágrafo de cada seção.
\usepackage{color}				% Controle das cores
\usepackage{graphicx}			% Inclusão de gráficos
\usepackage{microtype} 			% para melhorias de justificação
\usepackage{placeins}
\usepackage{bigfoot} % to allow verbatim in footnote
\usepackage[numbered,framed]{matlab-prettifier}

% ---

% ---
% Pacotes adicionais, usados no anexo do modelo de folha de identificação
% ---
\usepackage{multicol}
\usepackage{multirow}
% ---

% ---
% Pacotes adicionais, usados apenas no âmbito do Modelo Canônico do abnteX2
% ---
\usepackage{lipsum}				% para geração de dummy text
% ---

% ---
% Pacotes de citações
% ---
\usepackage[brazilian,hyperpageref]{backref}	 % Paginas com as citações na bibl
\usepackage[alf]{abntex2cite}	% Citações padrão ABNT

% --- 
% CONFIGURAÇÕES DE PACOTES
% --- 

% ---
% Configurações do pacote backref
% Usado sem a opção hyperpageref de backref
\renewcommand{\backrefpagesname}{Citado na(s) página(s):~}
% Texto padrão antes do número das páginas
\renewcommand{\backref}{}
% Define os textos da citação
\renewcommand*{\backrefalt}[4]{
	\ifcase #1 %
	Nenhuma citação no texto.%
	\or
	Citado na página #2.%
	\else
	Citado #1 vezes nas páginas #2.%
	\fi}%
% ---
%redefine a capa
\renewcommand{\imprimircapa}{%
	\begin{capa}%
		\center
		\ABNTEXchapterfont\Large \textbf{UNIVERSIDADE FEDERAL DE SERGIPE}
		\\
		\vspace*{1cm}
		{\ABNTEXchapterfont\large\imprimirautor}
		\vfill
		\begin{center}
			\ABNTEXchapterfont\bfseries\LARGE\imprimirtitulo
		\end{center}
		\vfill
		\large\imprimirlocal \\
		\large\imprimirdata
		\vspace*{1cm}
	\end{capa}
}
% ---
% Informações de dados para CAPA e FOLHA DE ROSTO
% ---
\titulo{Aplicação dos Algoritmos de Dijkstra e de Floyd-Warshall no software MATLAB}
\autor{Iuri Rodrigo Ferreira Alves da silva\\Gregory Medeiros Melgaço Pereira\\Raul Rodrigo Silva de Andrade \\ Rafael Castro Nunes \\ Ruan Robert Bispo dos Santos \\ Vítor do Bomfim Almeida Carvalho}
\local{São Cristóvão,SE}
\data{\today}
\instituicao{%
	Universidade Federal De Sergipe
	\par
	Faculdade de Engenharia Eletrônica
	\par
	Redes e Comunicações}
\tipotrabalho{Relatório técnico}
% O preambulo deve conter o tipo do trabalho, o objetivo, 
% o nome da instituição e a área de concentração 
\preambulo{Relatório em conformidade com as normas ABNT}
% ---

% ---
% Configurações de aparência do PDF final

% alterando o aspecto da cor azul
\definecolor{blue}{RGB}{41,5,195}

% informações do PDF
\makeatletter
\hypersetup{
	%pagebackref=true,
	pdftitle={\@title}, 
	pdfauthor={\@author},
	pdfsubject={\imprimirpreambulo},
	pdfcreator={LaTeX with abnTeX2},
	pdfkeywords={abnt}{latex}{abntex}{abntex2}{relatório técnico}, 
	colorlinks=true,       		% false: boxed links; true: colored links
	linkcolor=blue,          	% color of internal links
	citecolor=blue,        		% color of links to bibliography
	filecolor=magenta,      		% color of file links
	urlcolor=blue,
	bookmarksdepth=4
}
\makeatother
% --- 

% --- 
% Espaçamentos entre linhas e parágrafos 
% --- 

% O tamanho do parágrafo é dado por:
\setlength{\parindent}{1.3cm}

% Controle do espaçamento entre um parágrafo e outro:
\setlength{\parskip}{0.2cm}  % tente também \onelineskip

% ---
% compila o indice
% ---
\makeindex
% ---

% ----
% Início do documento
% ----
\begin{document}
	\lstset{language=Matlab} 
	% Seleciona o idioma do documento (conforme pacotes do babel)
	%\selectlanguage{english}
	\selectlanguage{brazil}
	
	% Retira espaço extra obsoleto entre as frases.
	\frenchspacing 
	
	% ----------------------------------------------------------
	% ELEMENTOS PRÉ-TEXTUAIS
	% ----------------------------------------------------------
	
	% ---
	% Capa
	% ---
	\imprimircapa
	% ---
	
	% ---
	% Folha de rosto
	% (o * indica que haverá a ficha bibliográfica)
	% ---
	\imprimirfolhaderosto*
	% ---
	
	% ---
	% Agradecimentos
	% ---
	
	% ---
	
	% RESUMO
	% resumo na língua vernácula (obrigatório)
	\setlength{\absparsep}{18pt} % ajusta o espaçamento dos parágrafos do resumo
	\begin{resumo}
		
		A busca por uma maior economia e menores custos nas diversas áreas das ações humanas foi um grande ponto de partida para o surgimento da Teoria dos Grafos com o consequente Problema dos Menores Caminhos. Para a resolução desse Problema dos Menores Caminhos surgiram vários tipos de algoritmos, como por exemplo o Algoritmo de Dijkstra e Algoritmo de Floyd-Warshall.
		
		 Nesse trabalho será feito uma explicação sobre o que são os grafos e sobre os algoritmos utilizados nesse projeto. Após isso, esses algoritmos serão implementados no software computacional MATLAB para ser possível a comparação entre as características obtidas dos dois algoritmos. Com essa comparação será possível dizer qual dos algoritmos é mais eficiente com relação ao número de iterações e tempo de processamento e sob quais condições isso acontece.
		
		\noindent
		\textbf{Palavras-chaves}: Grafos, Problema dos Menores Caminhos, Floyd-Warshall, Dijkstra, Iterações
	\end{resumo}
	% ---
	
	%lista de ilustrações
	% ---
	\pdfbookmark[0]{\listfigurename}{lof}
	\listoffigures*
	\pagebreak
	% ---
	
	%lista de tabelas
	\pdfbookmark[0]{\listtablename}{lot}
	\listoftables*
	\pagebreak
	% ---
	
	%lista de abreviaturas e siglas
	
	
	% sumario
	\pdfbookmark[0]{\contentsname}{toc}
	\tableofcontents*
	\pagebreak
	% ---
	
	%introdução
	\chapter*[Introdução]{Introdução}
\addcontentsline{toc}{chapter}{Introdução}

No mundo globalizado e capitalista que vivemos é cada vez mais necessário caminhos ou métodos para se decidir qual a melhor alternativa ou caminho a ser usado visando uma maior economia, , seja ela na produção do meio industrial, nas conversões de moedas ou até melhores caminhos a serem tomados numa viagem. 

Foi pensando nesses tipos de problemas que surgiu a Teoria dos Grafos em 1736 com Leonhard Euler (1707 – 1783). Um grafo é um conjunto de vértices e um conjunto de arestas que ligam pares de vértices distintos (com nunca mais que uma aresta a ligar qualquer par de vértices).

  Leonhard resolveu um quebra-cabeça local da cidade Königsberg onde vivia, hoje conhecido como O Problema das Pontes de Königsberg. Esse problema consistia em partindo de um ponto inicial, voltar a esse mesmo ponto passando exatamente uma vez por cada ponte da cidade. Euler mostrou que isso só seria possível se cada porção de terra estivesse ligada a um número par de pontes, e , como isso não acontecia na sua cidade, não era possível.
  
    Outro exemplo nesse mesmo sentido é o mostrado em \cite{costalonga2012grafos}, em que um carteiro sempre tentará caminhar a menor distancia possível, assim, para isso acontecer, ele tem de agir de tal forma a passar um número mínimo de vezes em determinada rua, otimizando assim seu trabalho.
  
   Ao longo da história outros problemas similares ao de Euler surgiram, como por exemplo ‘O Problema das Quatro Cores’ . Esse problema surgiu em 1852 e somente em 1976 utilizando métodos computacionais chegou-se a sua solução. 

A Teoria dos Grafos ganhou muita visibilidade já no século passado, pois como dito em \cite{da2011teoria} no desenvolvimento matemático  assim como nas suas aplicações, foi dada grande importância a essa teoria. Com o passar do tempo grafos foram utilizados em várias áreas do conhecimento, desde a Economia até a Biologia. 

Com esse grande destaque, se tornou necessária a criação de algoritmos que agilizassem suas soluções. Entre os mais conhecidos estão o de Dijkstra, de Bellman-Ford e o de Floyd-Warshall.

	
	
	
	\phantompart
	%-
	
	%Revisão Bibliografica
	\include{Referência}
	\phantompart
	%-
	\chapter{Objetivos}
	
	O objetivo do trabalho foi o estudo da teoria dos grafos na resolução do problema de menor caminho. Para tal, foram implemantados os algoritmos de Dijkstra e de Floyd-Warshall com objetivo de comparar o desempenho dos mesmos na resolução do problema.
	
	\include{teorica}
	\phantompart
	\chapter{Formulação do problema}
	
	Os algoritmos foram implementados no software MATLAB. Seguem abaixo os códigos comentados :
	
	Grafos:
	
	\begin{lstlisting}[ basicstyle=\tiny,frame=single]
	
	clc 
	clear
	hold off
	
	% Define ligações
	s = [1 1 1 2 2 3 3 4 5 5 6 7]; 
	t = [2 4 8 3 7 4 6 5 6 8 7 8]; 
	
	% q = [1 1 3 3 5 5 7 7 9 9 11 11 13
	 13 15 15 17 17 19 19 2 6 10 
	14 18 4 8 12 16 20];
	
	% e = [2 3 4 5 6 7 8 9 10 11 12 13
	 14 15 16 17 18 19 20 1 6 10 14 
	18 2 8 12 16 20 4];
	
	% Pesos
	weights = [10 10 1 10 1 10 1 1 12 12 12 12];
	% weights = [10 10 1 10 1 10 1 1 
	12 12 12 12 9 2 5 7 12 14 9 1 3 2
	24 12 14 10 12 1 2 3];
	
	% weights = [10 10 1 10 10 10 1 10 1 10 1 1
	 12 12 12 12 9 2 5 7 12 14 9 1 3 2 24 12 14
	  10 12 1 2 3 1 10 1 1 10 10 1 10 1 10 1 1 
	  12 12 12 12 9 2 5 7 12 14 9 1 3 2 24 12 
	  14 10 12 1 2 3 12 12 12 12 9 2 5 7 12 14 
	  9 1 3 2 24 12 14 10 12 1 2 3];
	
	% Nomes
	names = {'A' 'B' 'C' 'D' 'E' 'F' 'G' 'H'};
	
	% Cria Grafo
	G = graph(s,t,weights,names);
	% G = graph(q,e,weights);
	% G = graph(bucky);
	
	% Plots
	p = plot(G,'EdgeLabel',G.Edges.Weight);
	
	% Menor distancia
	tic
	[caminho,distancia] = dijkstra(G,2,5);
	Td = toc;
	tic
	[caminho2,distancia2] = floyd(G,2,5);
	Tf = toc;
	
	% Plot fresco
	pause(1);
	duo = caminho(1);
	for i=1:(size(caminho,2)-1)
	duo = [duo caminho(i+1)];
	highlight(p, duo,'EdgeColor','r')
	pause(1);
	end
	
	figure(2)
	c = categorical({'Floyd','Dijkstra'});
	v = [Tf Td];
	bar(c,v)
	
	\end{lstlisting}
	Dijkstra:
	
	\begin{lstlisting}[ basicstyle=\tiny,frame=single]
	
function [Caminho, Distancia] = dijkstra(G, origem, destino)
	
% Entrada: Grafo G
%          Origem da busca
%          Destino da busca
% Saída  : Um menor Caminho
%          A Distancia mínima
	
Caminho = [destino]; % inicializa o vetor caminho
Distancia = []; % inicializa o vetor distancia
	
% criando a matriz dados
dist = (inf*(1:size(G.Nodes,1))'); %distancias infinito
preced = (0*(1:size(G.Nodes,1))'); % coluna que indica de onde veio 
dados = [(1:size(G.Nodes,1))' dist preced]; %nó, distancia,origem
	
dados(origem,2) = 0; % distancia da origem a origem é zero
P_no = origem; % proximo no recebe origem (inicializacao)
memo = []; % memoria de Nós percorridos
	
while (1)
	
atual = P_no; % Nó atual
memo = [memo; dados(atual,:)]; % por onde ja passei
	
vizinhos = neighbors(G,atual); % acha Nós vizinhos
	
for i=1:size(vizinhos,1) % atribuir distancias
if (dados(vizinhos(i),2) > dados(atual,2)+distances(G,atual,vizinhos(i)));
dados(vizinhos(i),2) = dados(atual,2)+distances(G,atual,vizinhos(i));
dados(vizinhos(i),3) = dados(atual,1);
end
end
	
% Define o proximo Nó da busca
vazio = dados; % cria uma matriz que pode ser alterada
for l=1:size(memo,1)
ind = find(vazio == memo(l,1)); % encontra elementos ja visitados
vazio(ind(1),:) = []; % elimina elementos passados
end
[x,y] = min(vazio,[],1);
%-----------------------------%
% criterio de parada
if (isempty(vazio))
break
else
P_no = vazio(y(2),1); % escolhe proximo nó
end
end
	
% Define menor caminho
g = destino;
while (g ~= origem)
o = find(memo == g);
g = memo(o(1),3);
Caminho = [Caminho g];
end
Caminho = flip(Caminho);
	
% Define a menor Distancia
o = find(memo == destino);
Distancia = memo(o(1),2);
	
	
end
	\end{lstlisting}
	
	Floyd-Warshall:
	
	\begin{lstlisting}[ basicstyle=\tiny,frame=single]
	function [Caminho, Distancia] = floyd(G, origem, destino)
	
	% Entrada: Grafo G
	%          Origem da busca
	%          Destino da busca
	% Saída  : Um menor Caminho
	%          A Distancia mínima
	
	Distancia = []; % inicializa o vetor distancia
	Caminho = destino; % inicializa o vetor caminho
	
	m = [0    10   Inf    10   Inf   Inf   Inf     1
	10     0    10   Inf   Inf   Inf     1   Inf
	Inf    10     0    10   Inf     1   Inf   Inf
	10   Inf    10     0     1   Inf   Inf   Inf
	Inf   Inf   Inf     1     0    12   Inf    12
	Inf   Inf     1   Inf    12     0    12   Inf
	Inf     1   Inf   Inf   Inf    12     0    12
	1   Inf   Inf   Inf    12   Inf    12     0];
	dis = m;
	
	cam = [ 1 2 3 4 5 6 7 8;
	1 2 3 4 5 6 7 8;
	1 2 3 4 5 6 7 8;
	1 2 3 4 5 6 7 8;
	1 2 3 4 5 6 7 8;
	1 2 3 4 5 6 7 8;
	1 2 3 4 5 6 7 8;
	1 2 3 4 5 6 7 8];
	
	for k=1:size(m,1)
	for i=1:size(m,1)
	for j=1:size(m,1)
	if (dis(i,k)+ dis(k,j) < dis(i,j))
	cam(i,j)=k;
	dis(i,j) = dis(i,k)+ dis(k,j);
	end
	end
	end
	end
	
	Distancia = dis(origem,destino);
	
	while(Caminho(size(Caminho,1)) ~= origem)
	Caminho = [Caminho; cam(destino,origem)];
	if Caminho(size(Caminho,1)) ~= origem
	destino = Caminho(size(Caminho,1));
	end
	end
	
	Caminho = flip(Caminho');
	
	end
	
	\end{lstlisting}
	
	\phantompart
	\chapter{Resultados Obtidos}
	Como resultados, os dois códigos feitos no MATLAB funcionaram perfeitamente. Eles foram capazes de dizer a melhor rota como também o custo dessa rota. Já na comparação entre os dois desepenhos o algoritmo de Floyd-Warshall possuiu um tempo de processamento muito menor do que o tempo de processamento do Dijkstra. Essa diferença nos tempos pode ser visualizada no gráfico de barras abaixo, onde a coluna da esquerda(verde) apresenta o tempo do Dijkstra e o da direita(azul) é o tempo do Floyd-Warshall :
	
	\begin{figure}[h!]
		\centering
		\includegraphics[scale=0.7]{img5.jpg}
		\caption{Gráfico de barras Dijkstra e Floyd}
		\label{img5}
	\end{figure}
	
	\begin{figure}[h!]
		\centering
		\includegraphics[scale=0.5]{img6.jpg}
		\caption{Grafo com caminho obtido}
		\label{img6}
	\end{figure}
	\FloatBarrier
  Isso já era esperado pois no algoritmo de Dijkstra o número de iterações é muito superior ao do Floyd-Warhsall já que a cada vértice ele precisa analisar o caminho a ser tomado. No algoritmo de Floyd-Warshall é menor, onde número de iterações é igual ao número de vértices do grafo.
  
  A figura \ref{img7} apresenta novamente uma comparação de desempenho para o grafo da figura \ref{img8}, mostrando o mesmo resultado superior do floyd sobre o Dijkstra.
  
  \begin{figure}[h!]
  	\centering
  	\includegraphics[scale=0.7]{img7.jpg}
  	\caption{Gráfico de barras Dijkstra e Floyd}
  	\label{img7}
  \end{figure}

	\begin{figure}[h!]
		\centering
		\includegraphics[scale=0.5]{img8.jpg}
		\caption{Grafo com caminho obtido}
		\label{img8}
	\end{figure}
  	
  	\FloatBarrier
  	
  	Foi notado que a partir de 700 vértices, o algorítimo de Dijkstra apresentou um desempenho superior ao de Floyd-Warshall.
	\phantompart
	\chapter{Conclusão}
	
	\phantompart
	
	
	\bibliography{biblio}
	%\bibliographystyle{ieeetr}
\end{document}
